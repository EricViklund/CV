\documentclass[]{revtex4-2}
\usepackage{siunitx}
\usepackage{graphicx}

\begin{document}

\title{Research Statement}
\author{Eric Viklund}
\email{ericviklund2023@u.northwestern.edu}
\affiliation{Department of Materials Science and Engineering, Northwestern University}
\affiliation{Fermi National Accelerator Laboratory}


\date{\today}


\maketitle


\section{Research Experience}

    My research as a PhD student in the Northwestern University Materials Science and Engineering department was funded by a collaboration between Fermi National Accelerator Laboratory (FNAL), organized by Sam Posen, and my advisor professor David Seidman. Through this collaboration I have gained an extensive knowledge of materials science and metallurgy in addition to experience with accelerator science and superconducting radiofrequency (SRF) cavity technology. 

    My work at FNAL has led to several developments in manufacturing capabilities for Nb\textsubscript{3}Sn SRF cavities. These capabilities include the development of a new mechanical polishing technique for Nb\textsubscript{3}Sn cavities, which produces smooth Nb\textsubscript{3}Sn films that were previously unobtainable using chemical polishing methods.\cite{10.1088/1361-6668/ad0b2d} Nb\textsubscript{3}Sn films treated with mechanical polishing have a mirror-like finish which cannot be obtained using other polishing methods. This improvement in surface roughness led to a significant increase in the maximum accelerating gradient of a Nb\textsubscript{3}Sn cavity.
    
    In addition to mechanical polishing, I have also worked on a new Sn coating procedure, which has been used to heal Nb\textsubscript{3}Sn cavities that have suffered degradation due to excess stresses applied to the cavity. Due to the thin and brittle nature of the Nb\textsubscript{3}Sn films used in Nb\textsubscript{3}Sn cavities, they are susceptible to developing cracks during handling and tuning. Using a short, low-temperature Sn coating, a large percentage of the performance of a stress degraded cavity was recovered.

    In addition to my work on Nb\textsubscript{3}Sn, I have also dedicated a lot of time to understand electropolishing (EP). This process, which is applied to Nb SRF cavities has been a large driver in cavity performance improvements over the past several decades. However, our fundamental understanding of the physical mechanisms of this process are still lacking. By systematically polishing Nb samples at different voltages and temperatures I was able to show that etching is caused by insufficient voltage applied to the sample, and is not significantly affected by the electrolyte temperature or the presence of nitrogen doping in the Nb.\cite{viklund:srf2021-supcav016}

    Through my research, I have become an expert in various measurement techniques for surface characterization. Electron microscopy, both transmission and scanning, has been an important tool for my research since the beginning. Mulitple imaging techniques, such as electron dispersive x-ray spectroscopy and electron backscatter diffraction, were utilized to analyze my samples. I am also an expert in using dual beam FIB/SEM instruments to prepare samples such as microtips for atom probe tomography or lamella for TEM. Additionally, I have used a dual beam FIB/SEM to perform three-dimensional EDS measurements, which involves the automated serial sectioning of a sample utilizing a focussed ion beam followed by imaging using EDS to obtain a 3D view of the sample composition. I used this method to image Sn-deficient regions in Nb\textsubscript{3}Sn thin films which are a major source of performance degradation in Nb\textsubscript{3}Sn cavities. \cite{viklund2023three} Apart from electron microscopy, I also have experience with many other analysis techniques such as laser confocal microscopy, secondary ion mass spectroscopy (SIMS), atomic force microscopy (AFM), and x-ray photoelectron spectroscopy (XPS).

    I have presented my research at many conferences such as the International Conference on Radio-Frequency Superconductivity (SRF), the International Particle Accelerator Conference (IPAC), the Applied Superconductivity Conference (ASC), and the Cryogenic Engineering Conference and International Cryogenic Materials Conference (CEC/ICMC). I was a recipient for the IPAC'23 Student Grant, and I am a part of the Fermilab Accelerator PhD Program, which provides me with a grant for my tuition and stipend.







\newpage

\bibliographystyle{plain}
\bibliography{ref.bib}

\end{document}