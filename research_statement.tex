\documentclass[]{revtex4-2}
\usepackage{siunitx}
\usepackage{graphicx}

\begin{document}

\title{Research Statement}
\author{Eric Viklund}
\email{ericviklund2023@u.northwestern.edu}
\affiliation{Department of Materials Science and Engineering, Northwestern University}
\affiliation{Fermi National Accelerator Laboratory}


\date{\today}

\begin{abstract}



\end{abstract}

\maketitle

\section{Research Experience}

    My research as a graduate student for the past 4 years has been broadly focussed on making  developments in superconducting radiofrequency (SRF) cavity technology. My work at Fermilab, under the guidance of Sam Posen, has led to several developments in our manufacturing capabilities for Nb\textsubscript{3}Sn SRF cavities. These capabilities include the development of a new mechanical polishing technique for Nb\textsubscript{3}Sn cavities, which has allowed us to produce smooth Nb\textsubscript{3}Sn films that were previously unobtainable using chemical polishing methods.\cite{viklund2023improving} The capabilities and impact of this new method are still being explored, but promising results have already begun to materialize. In addition to this polishing method, I have also worked on a new Sn coating procedure, which seems to be useful for healing Nb\textsubscript{3}Sn cavities that have suffered degradation due to excess stresses on the cavity. Due to the thin and brittle nature of the Nb\textsubscript{3}Sn films used in Nb\textsubscript{3}Sn cavities, they are suseptible to cracking during handling and tuning. Using a short, low-temperature Sn coating, a large percentage of cavity performance of a stress degraded cavity was recovered. The procedure was originally used to heal damage caused by the aforementioned mechanical polishing process, but also seems to be very useful in this situation.

    In addition to my work on Nb\textsubscript{3}Sn, I have also dedicated a lot of time to understand electropolishing (EP). This process, which is applied to Nb SRF cavities has been a large driver in cavity performance improvements over the past several decades. However, our fundamental understanding of the chemical-electro-hydrodynamic mechanisms of this process are still lacking. By systematically polishing Nb samples at different voltages and temperatures I was able to show that etching is caused by insufficient voltage applied to the sample, and is not significantly affected by the electrolyte temperature or the presence of nitrogen doping in the Nb.\cite{viklund:srf2021-supcav016}

    Through my research, I have become an expert in various measurement techniques for surface characterization. I am particularly well versed in electron microscopy. I have experience with both transmission and scanning electron microscopy\cite{viklund2023improving} as well as mulitple imaging techniques such as electron dispersive x-ray spectroscopy and electron backscatter diffraction. I am also an expert in using dual beam FIB/SEM instruments to prepare samples such as microtips for atom probe tomography or lamella for TEM. Additionally, I have used a dual beam SEM to perform three-dimensional EDS measurements, which involves the automated serial sectioning and imaging of a 3D volume. I used this method to image Sn-deficient regions in Nb\textsubscript{3}Sn thin films. \cite{viklund2023three} Apart from electron microscopy, I also have experience with many other analysis techniques such as laser confocal microscopy, secondary ion mass spectroscopy (SIMS), atomic force microscopy (AFM), and x-ray photoelectron spectroscopy (XPS).

    I have presented my research at many different conferences such as the International Conference on Radio-Frequency Superconductivity (SRF), the International Particle Accelerator Conference (IPAC), the Applied Superconductivity Conference (ASC), and the Cryogenic Engineering Conference and International Cryogenic Materials Conference (CEC/ICMC). I was a recipient for the IPAC'23 Student Grant and I am a part of the Fermilab Accelerator PhD Program, which provides me with a grant for my tuition and stipend.





\section{Research Plan}


    Thanks to my research experience in SRF cavities, I am a well suited candidate to continue the world-leading cost reduction research efforts for the International Linear Collider project at KEK. There are three research topics that I am particularly qualified to explore at KEK relating to ILC cost reduction research: cost reductions in electropolishing, and development of medium and large grain Nb SRF cavities.
    
    Thanks to my first hand experience of working with electropolishing, I am well aware of the costs, both monetary and environmental, of performing electropolishing on Nb SRF cavities using HF containing electrolyte. HF presents a significant danger to the people who work with it on an industrial scale. This inherent danger complicates the Nb cavity production process and increases the cost of the cavities. In order to meet the production target of around 900 cryo-modules required for the ILC, the electropolishing process must be improved to allow for greater scalability.

    One way to achieve this is to eliminate the use of HF or any other hazardous chemical. There are several technologies available or in development that could achieve this such as plasma electropolishing (PEP), centrifugal barrel polishing (CBP), and bipolar electropolishing (BPEP). Of these, CBP and pulse reverse electropolishing are the most mature.

    Typically, centrifugal barrel polished cavities must go though an additional HF-based electropolishing step in order to remove embedded abrasives from the CBP step. This eliminates the main benefit of using CBP, which is the elimination of HF. However, in my research on mechanically polishing Nb\textsubscript{3}Sn cavities I have shown definitively that good performance can be obtained without the need of an additional electropolishing step. My research goal at KEK will be to apply this process to Nb cavities to see if high performance is possible without the need of an electropolishing step.

    Bipolar electropolishing has been shown to be effective for polishing Nb SRF cavities. However, there is still work to be done to improve the technological readiness of the process to a point where it can be used for mass production of SRF cavities. Thanks to my extensive experience working with electropolishing I am well qualified to study this new and exciting technique. I plan to use my knowledge of electrochemistry to optimize the removal rate and surface finish of the BPEP technique. 

\bibliographystyle{plain}
\bibliography{ref.bib}