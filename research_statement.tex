\documentclass[]{revtex4-2}
\usepackage{siunitx}
\usepackage{graphicx}

\begin{document}

\title{Research Statement}
\author{Eric Viklund}
\email{ericviklund2023@u.northwestern.edu}
\affiliation{Department of Materials Science and Engineering, Northwestern University}
\affiliation{Fermi National Accelerator Laboratory}


\date{\today}


\maketitle

\section{Introduction}

    Here I will present my plan for research towards SRF cavity cost reduction for the ILC project as an assistant professor at KEK. I will explain my past research and experience in the field of SRF and accelerator science. This includes research into electropolishing using advanced electrochemical techniques such as electrochemical impedance spectroscopy, and research into new methods of polishing Nb\textsubscript{3}Sn SRF cavities using centrifugal barrel polishing. This experience makes me well qualified for performing research at KEK.

    I will also propose a new research project that will help improve our ability for large scale niobium SRF cavity production for the ILC project. Using my experience in electrochemistry, I plan to study a new method of electropolishing SRF cavities known as bipolar electropolishing which does not require the use of toxic HF acid. This research goal is well aligned with KEK's initiative to reduce the manufacturing and environmental costs of ILC construction by eliminiating the safety and environmental hazards of using HF acid.  

\section{Research Experience}

    My research as a PhD student in the Northwestern University Materials Science and Engineering department was funded by a collaboration between Fermi National Accelerator Laboratory (FNAL), organized by Sam Posen, and my advisor professor David Seidman. Through this collaboration I have gained an extensive knowledge of materials science and metallurgy in addition to experience with accelerator science and superconducting radiofrequency (SRF) cavity technology. 

    My work at FNAL has led to several developments in manufacturing capabilities for Nb\textsubscript{3}Sn SRF cavities. These capabilities include the development of a new mechanical polishing technique for Nb\textsubscript{3}Sn cavities, which produces smooth Nb\textsubscript{3}Sn films that were previously unobtainable using chemical polishing methods.\cite{viklund2024improving} Nb\textsubscript{3}Sn films treated with mechanical polishing have a mirror-like finish which cannot be obtained using other polishing methods. This improvement in surface roughness led to a significant increase in the maximum accelerating gradient of a Nb\textsubscript{3}Sn cavity.
    
    In addition to mechanical polishing, I have also worked on a new Sn coating procedure, which has been used to heal Nb\textsubscript{3}Sn cavities that have suffered degradation due to excess stresses applied to the cavity. Due to the thin and brittle nature of the Nb\textsubscript{3}Sn films used in Nb\textsubscript{3}Sn cavities, they are susceptible to developing cracks during handling and tuning. Using a short, low-temperature Sn coating, a large percentage of the performance of a stress degraded cavity was recovered.

    In addition to my work on Nb\textsubscript{3}Sn, I have also dedicated a lot of time to understand electropolishing (EP). This process, which is applied to Nb SRF cavities has been a large driver in cavity performance improvements over the past several decades. However, our fundamental understanding of the physical mechanisms of this process are still lacking. By systematically polishing Nb samples at different voltages and temperatures I was able to show that etching is caused by insufficient voltage applied to the sample, and is not significantly affected by the electrolyte temperature or the presence of nitrogen doping in the Nb.\cite{viklund:srf2021-supcav016}

    Through my research, I have become an expert in various measurement techniques for surface characterization. Electron microscopy, both transmission and scanning, has been an important tool for my research since the beginning. Mulitple imaging techniques, such as electron dispersive x-ray spectroscopy and electron backscatter diffraction, were utilized to analyze my samples. I am also an expert in using dual beam FIB/SEM instruments to prepare samples such as microtips for atom probe tomography or lamella for TEM. Additionally, I have used a dual beam FIB/SEM to perform three-dimensional EDS measurements, which involves the automated serial sectioning of a sample utilizing a focussed ion beam followed by imaging using EDS to obtain a 3D view of the sample composition. I used this method to image Sn-deficient regions in Nb\textsubscript{3}Sn thin films which are a major source of performance degradation in Nb\textsubscript{3}Sn cavities. \cite{viklund2023three} Apart from electron microscopy, I also have experience with many other analysis techniques such as laser confocal microscopy, secondary ion mass spectroscopy (SIMS), atomic force microscopy (AFM), and x-ray photoelectron spectroscopy (XPS).

    I have presented my research at many conferences such as the International Conference on Radio-Frequency Superconductivity (SRF), the International Particle Accelerator Conference (IPAC), the Applied Superconductivity Conference (ASC), and the Cryogenic Engineering Conference and International Cryogenic Materials Conference (CEC/ICMC). I was a recipient for the IPAC'23 Student Grant, and I am a part of the Fermilab Accelerator PhD Program, which provides me with a grant for my tuition and stipend.





\section{Research Plan}

    Thanks to my first-hand experience of working with electropolishing, I am well aware of the costs, both monetary and environmental, of performing EP on Nb SRF cavities using HF containing electrolyte. HF presents a significant danger to the environment and to the people who work with it on an industrial scale, which complicates the Nb cavity EP process and increases the cost of the cavities. In order to meet the production target of around 900 cryo-modules required for the ILC, the EP process must be improved to allow for greater scalability.

    One way to achieve this is to eliminate the use of HF and other hazardous chemical from the EP process. There are several technologies available to achieve this such as plasma electropolishing (PEP), centrifugal barrel polishing (CBP), and bipolar electropolishing (BPEP). Of these, BPEP is the most mature for replacing traditional chemical polishing methods. Multicell bipolar EP capabilities are already under development at Jefferson Lab\cite{tian:srf2019-frcab1} and at Oakridge National Lab\cite{inman2015electropolishing}.
    
    To make BPEP a viable alternative to traditional electropolishing techniques for ILC cavity production we first need to gain a better understanding of the process. BPEP uses alternating anodic and cathodic voltage pulses. The anodic pulse is used to oxidize the niobium surface and the cathodic pulse then dissolves the oxide. However, the mechanism causing the oxide dissolution is still not known. Most likely the cathodic pulse is causing corrosion driven by hydrolysis. The removal rate of BPEP has been shown to depend only on the voltage of the cathodic pulse and not on the anodic pulse, which means that a better understanding of the cathodic pulse is necessary to optimize the BPEP process.

    To study cathodic pulse, I will use electrochemical impedance spectroscopy (EIS) and cyclic voltametry (CV) to characterize the cathodic pulse reaction. These electrochemical methods can give important information about the cathodic reaction such as the rate parameters of reaction steps during the cathodic pulse. Using these techniques together with other surface microscopy of samples polishing using BPEP, I will develop a model for the cathodic dissolution mechanism. 

    The second step towards making BPEP a viable alternative to traditional EP is the development of a pulsed power supply capable of providing very high power and repetition rates for multicell cavities. To maximize the removal rate, a high current is necessary due to the large surface area of multicell cavities. Additionally, a repetition rate of around 60~Hz or higher is needed to ensure a high removal rate. This has been achieved using a high power DC power supply together with capacitor banks and a switching circuit to generate the pulses. There is also the option of using an alternating power supply which are much simpler in design and are readily available commercially. However, so far there has not been any attempts to use this type of power supply since they can only output a sinusoidal waveform. 

    KEK is well suited to conduct research on BPEP thanks to its extensive record of studying and using EP for SRF cavity production. Additionally, KEK has the ability to perform vertical EP, which is necessary for BPEP due to the low viscosity of the electrolyte.

\section{Conclusion}
    
    Throughout my time as a PhD student at Northwestern University and FNAL, I have become an expert in material analysis using techniques such as focussed ion beam and scanning electron microscopy, transmission electron microscopy, and many other techniques. I have also extensively studied electrochemistry to try understand the niobium electropolishing process. I have studied the chemistry of niobium EP using poteniometric methods as well as electrochemical impedance spectroscopy.

    Using this expertice, I plan to study bipolar electropolishing at KEK. This method can reduce the cost of cavity polishing by eliminating the need for toxic HF acid. This is necessary to ensure safety and environmental integrity as the production of SRF cavities is ramped up for the ILC project. My plan is to apply my knowledge of electrochemistry and advanced electrochemical analysis techniques to better understand the cathodic pulse oxide dissolution process which determines the removal rate and surface finish produced by BPEP. This will allow for a wider adoption of the technique in the SRF community reducing our reliance on HF acid. Using the vertical EP capabilities at KEK will be essential for my research, since BPEP has been shown to be more reliable using this method of EP.
    

\newpage

\bibliographystyle{plain}
\bibliography{ref.bib}

\end{document}