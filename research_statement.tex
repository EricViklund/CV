\documentclass[]{revtex4-2}
\usepackage{siunitx}
\usepackage{graphicx}

\begin{document}

\title{Research Statement}
\author{Eric Viklund}
\email{ericviklund2023@u.northwestern.edu}
\affiliation{Department of Materials Science and Engineering, Northwestern University}
\affiliation{Fermi National Accelerator Laboratory}


\date{\today}

\begin{abstract}



\end{abstract}

\maketitle

\section{Research Experience}

    My research as a graduate student for the past 4 years has been broadly focussed on making  developments in superconducting radiofrequency (SRF) cavity technology. My work at Fermilab, under the guidance of Sam Posen, has led to several developments in our manufacturing capabilities for Nb\textsubscript{3}Sn SRF cavities. These capabilities include the development of a new mechanical polishing technique for Nb\textsubscript{3}Sn cavities, which has allowed us to produce smooth Nb\textsubscript{3}Sn films that were previously unobtainable using chemical polishing methods.\cite{viklund2023improving} The capabilities and impact of this new method are still being explored, but promising results have already begun to materialize. In addition to this polishing method, I have also worked on a new Sn coating procedure, which seems to be useful for healing Nb\textsubscript{3}Sn cavities that have suffered degradation due to excess stresses on the cavity. Due to the thin and brittle nature of the Nb\textsubscript{3}Sn films used in Nb\textsubscript{3}Sn cavities, they are suseptible to cracking during handling and tuning. Using a short, low-temperature Sn coating, a large percentage of cavity performance of a stress degraded cavity was recovered. The procedure was originally used to heal damage caused by the aforementioned mechanical polishing process, but also seems to be very useful in this situation.

    In addition to my work on Nb\textsubscript{3}Sn, I have also dedicated a lot of time to understand electropolishing (EP). This process, which is applied to Nb SRF cavities has been a large driver in cavity performance improvements over the past several decades. However, our fundamental understanding of the chemical-electro-hydrodynamic mechanisms of this process are still lacking. By systematically polishing Nb samples at different voltages and temperatures I was able to show that etching is caused by insufficient voltage applied to the sample, and is not significantly affected by the electrolyte temperature or the presence of nitrogen doping in the Nb.\cite{viklund:srf2021-supcav016}

    Through my research, I have become an expert in various measurement techniques for surface characterization. I am particularly well versed in electron microscopy. I have experience with both transmission and scanning electron microscopy\cite{viklund2023improving} as well as mulitple imaging techniques such as electron dispersive x-ray spectroscopy and electron backscatter diffraction. I am also an expert in using dual beam FIB/SEM instruments to prepare samples such as microtips for atom probe tomography or lamella for TEM. Additionally, I have used a dual beam SEM to perform three-dimensional EDS measurements, which involves the automated serial sectioning and imaging of a 3D volume. I used this method to image Sn-deficient regions in Nb\textsubscript{3}Sn thin films. \cite{viklund2023three} Apart from electron microscopy, I also have experience with many other analysis techniques such as laser confocal microscopy, secondary ion mass spectroscopy (SIMS), atomic force microscopy (AFM), and x-ray photoelectron spectroscopy (XPS).

    I have presented my research at many different conferences such as the International Conference on Radio-Frequency Superconductivity (SRF), the International Particle Accelerator Conference (IPAC), the Applied Superconductivity Conference (ASC), and the Cryogenic Engineering Conference and International Cryogenic Materials Conference (CEC/ICMC). I was a recipient for the IPAC'23 Student Grant and I am a part of the Fermilab Accelerator PhD Program, which provides me with a grant for my tuition and stipend.





\section{Research Plan}


    Thanks to my research experience in SRF cavities, I am a well suited candidate to continue the world-leading cost reduction research efforts for the International Linear Collider project at KEK. Thanks to the unique capabilities available at KEK there are several research topics that I can explore: low cost HF-free bipolar electropolishing, and the fabrication of medium and large grain Nb SRF cavities. With my expertice in material science and electrochemistry, I can make a significant contribution to these topics.
    

\subsection{Low Cost HF-free Bipolar Electropolishing}

    Thanks to my first hand experience of working with electropolishing, I am well aware of the costs, both monetary and environmental, of performing electropolishing on Nb SRF cavities using HF containing electrolyte. HF presents a significant danger to the environment and to the people who work with it on an industrial scale. This danger complicates the Nb cavity electropolishing process and increases the cost of the cavities. In order to meet the production target of around 900 cryo-modules required for the ILC, the electropolishing process must be improved to allow for greater scalability.

    One way to achieve this is to eliminate the use of HF or any other hazardous chemical from the electropolishing process. There are several technologies available or in development that could achieve this such as plasma electropolishing (PEP), centrifugal barrel polishing (CBP), and bipolar electropolishing (BPEP). Of these, BPEP seems the most promising for replacing traditional chemical polishing methods.

    Bipolar electropolishing has been shown to be effective for polishing Nb SRF cavities \cite{?}. However, there is still work to be done to improve the technological readiness of the process to a point where it can be used for mass production of SRF cavities. The main drawback of BPEP is the slow material removal rate. To be competitive with traditional EP, the removal rate must be at least as fast as cold EP, which is used to achieve the final, high quality surface finish. Currently, the BPEP method takes a whole day to remove a similar amount of material as traditional EP, which is not scalable for large volume cavity production.

    To make BPEP a viable alternative to EP, I will use my knowledge and experience in electrochemistry to optimize the polishing rate. The polishing rate is determined by the electrolyte and the voltage pulse waveform. The optimum electrolyte for BPEP is one that is close to neutral due to the safety and easy handling of such an electrolyte. The electrolyte should also be highly conductive to carry the large polishing currents of cavity polishing with minimal ohmic losses. The magnitude and duration of the voltage pulse must also be optimized to maximize the polishing current and create a smooth surface.

    

    Using electrochemical impedance spectroscopy, the different electrochemical processes can be differentiated and analyzed for different polishing conditions.

\subsection{Deformation and Annealing Study on Large Grain Niobium}

    I am very excited about the research efforts at KEK on the development of medium and large grain cavities. KEK is well equiped to for this type of research with the ability to perform almost every step of the manufacturing process in-house including cavity forming, welding, polishing, and heat treatment. To accomplish this research effort it is essential that we understand the mechanical properties of large grain niobium. Due to the anisotropic mechanical properties of large grain niobium, cavities manufactured using press forming methods frequently show defects such as cracks, dimensional deviations, and overall lower yield strength. These issues complicate the manufacturing process and make it difficult to comply with pressure vessel safety regulations. To make large grain cavities viable for the ILC project, these issues will need to be resolved. Here, I propose a modification to the commonly used press forming method for fabricating large grain cavities.

    Instead of forming the cavity in a single, large deformation step, the forming process can be split into multiple, progressive deformations. After each deformation the niobium is annealed at a high temperature. The annealing step removes statistical and geometrically necessary dislocations created during the deformation via recovery, annihilation of dislocation of opposite sign, and recrystallization, the creation of new grain boundaries by dislocation rearrangement. Annealing reduces the residual stresses in the material from the deformation and improves the ductility of the material for further deformation reducing the effects of different work hardening characteristics of different grain orientations. The final result is a more isotropic deformation and lower spring-back. By incorporating one or more recrystallization steps in the press forming process, the mechanical properties of the large grain niobium can be improved to eliminate dimensional deviations and comply with pressure vessel safety regulations.

    There have been several studies on the recrystallization characteristics of fine grain niobium, however the literature on large grain niobium recrystallization is not as extensive. My goal will be to study the recrystallization and deformation behavior of large grain niobium by  manufacturing large grain niobium half-cells using a progressive press forming method. I will utilize the press forming equipment at available at KEK's cavity fabrication facility to create several large grain niobium half-cells with different levels of deformation. These half-cells will be annealed in a heat treating furnace at \qty{900}{\degreeCelsius} or higher to activate the recrystallization process. I plan to investigate the grain texture and lattice distortion of the half cells after pressing and annealing using electron backscatter diffraction (EBSD). The dimensional accuracy and spring-back will be measured using a coordinate measuring machine (CMM). Development of the intermediate press forming dies will be labor-intensive, so an initial experiment using smaller single crystal or bi-crystal niobium mechanical testing samples will be used to investigate the effects of different grain orientations and accumulated strain on the recrystallization process. These samples can also provide more data on the anisotropy of the mechanical properties of niobium.
    
    This study will provide insight into the deformation mechanisms of large grain niobium and the recrystallization process leading to better large grain cavity dimensional accuracy and fewer manufacturing defects.





\bibliographystyle{plain}
\bibliography{ref.bib}

\end{document}