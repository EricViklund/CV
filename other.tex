Centrifugal barrel polishing is a very common choice for removing large amounts of material from the surface of cavities to remove defects present after cavity forming and welding, and the final surface finish is then obtained using traditional electropolishing or buffered chemical polishing. Typically, centrifugal barrel polished cavities must go though an additional HF-based electropolishing step in order to remove embedded abrasives from the CBP step, negating the advantage of HF free polishing. However, in my research on mechanically polishing Nb\textsubscript{3}Sn cavities I have shown that good performance can be obtained despite contamination from polishing media so long as the media is not conductive. This suggests that it is possible to polish Nb cavities using CBP without any additional chemical polishing. One of my research goals that I want to pursue at KEK is to achieve high quality factor and high gradient Nb SRF cavities using only CBP without the need for chemical polishing. 
    
To accomplish this goal, a new polishing recipe must be created. Typically a progression of different polishing media is used, starting with coarse grain media that maximize material removal rate and remove large scale imperfections then progressing to fine grain media that can remove small scale features and give the cavity a mirror finish. If good performance is to be achieved without the use of an electropolishing step, the type of media used will need to be chosen to eliminate any embedded abrasive particles and other contamination. To do this I will perform a study using a coupon cavity. This type of cavity can be used to hold small Nb samples and approximate the conditions on the cavity surface during CBP. This allows for rapid testing of different polishing media and conditions, which would otherwise be impossible when using regular cavities. These samples can then be analyzed to determine the surface finish and any contamination or embedded particles. Once a sufficient surface quality is achieved I will apply the process to a regular single cell cavity and measure the quality factor and accelerating gradient of the cavity using the vertical testing method.